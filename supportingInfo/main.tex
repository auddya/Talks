\documentclass{article}
\usepackage[utf8]{inputenc}
\newcommand{\grad}[1]{\gv{\nabla} #1}
%---------------------------------------------------------
%               Bold Face Math Italic:
%               All In Format: \b* .
%---------------------------------------------------------
\def\bA{\mbox{\boldmath$ A$}}
\def\bB{\mbox{\boldmath$ B$}}
\def\bC{\mbox{\boldmath$ C$}}
\def\bD{\mbox{\boldmath$ D$}}
\def\bE{\mbox{\boldmath$ E$}}
\def\bF{\mbox{\boldmath$ F$}}
\def\bG{\mbox{\boldmath$ G$}}
\def\bH{\mbox{\boldmath$ H$}}
\def\bI{\mbox{\boldmath$ I$}}
\def\bJ{\mbox{\boldmath$ J$}}
\def\bK{\mbox{\boldmath$ K$}}
\def\bL{\mbox{\boldmath$ L$}}
\def\bM{\mbox{\boldmath$ M$}}
\def\bN{\mbox{\boldmath$ N$}}
\def\bO{\mbox{\boldmath$ O$}}
\def\bP{\mbox{\boldmath$ P$}}
\def\bQ{\mbox{\boldmath$ Q$}}
\def\bR{\mbox{\boldmath$ R$}}
\def\bS{\mbox{\boldmath$ S$}}
\def\bT{\mbox{\boldmath$ T$}}
\def\bU{\mbox{\boldmath$ U$}}
\def\bV{\mbox{\boldmath$ V$}}
\def\bW{\mbox{\boldmath$ W$}}
\def\bX{\mbox{\boldmath$ X$}}
\def\bY{\mbox{\boldmath$ Y$}}
\def\bZ{\mbox{\boldmath$ Z$}}
\def\ba{\mbox{\boldmath$ a$}}
\def\bb{\mbox{\boldmath$ b$}}
\def\bc{\mbox{\boldmath$ c$}}
\def\bd{\mbox{\boldmath$ d$}}
\def\be{\mbox{\boldmath$ e$}}
\def\bff{\mbox{\boldmath$ f$}}
\def\bg{\mbox{\boldmath$ g$}}
\def\bh{\mbox{\boldmath$ h$}}
\def\bi{\mbox{\boldmath$ i$}}
\def\bj{\mbox{\boldmath$ j$}}
\def\bk{\mbox{\boldmath$ k$}}
\def\bl{\mbox{\boldmath$ l$}}
\def\bm{\mbox{\boldmath$ m$}}
\def\bn{\mbox{\boldmath$ n$}}
\def\bo{\mbox{\boldmath$ o$}}
\def\bp{\mbox{\boldmath$ p$}}
\def\bq{\mbox{\boldmath$ q$}}
\def\br{\mbox{\boldmath$ r$}}
\def\bs{\mbox{\boldmath$ s$}}
\def\bt{\mbox{\boldmath$ t$}}
\def\bu{\mbox{\boldmath$ u$}}
\def\bv{\mbox{\boldmath$ v$}}
\def\bw{\mbox{\boldmath$ w$}}
\def\bx{\mbox{\boldmath$ x$}}
\def\by{\mbox{\boldmath$ y$}}
\def\bz{\mbox{\boldmath$ z$}}

\title{A diffuse interface framework for modelling the evolution of multi-cell aggregates as a soft packing problem due to growth and division of cells}
\author{J. Jiang\thanks{Mechanical Engineering, University of Michigan}, K. Garikipati\thanks{Mechanical Engineering, and Mathematics, University of Michigan; corresponding author} \& S. Rudraraju\thanks{Mechanical Engineering, University of Wisconsin-Madison}}
\date

\usepackage{natbib}
\usepackage{graphicx}

\usepackage{amsmath}
\usepackage{bm}
\usepackage{bbm}
\usepackage{mathrsfs}
\usepackage{subcaption}
\usepackage{epsfig}
\usepackage{amsfonts}
\usepackage{amssymb}
\usepackage{wrapfig}
\usepackage{psfrag}
\usepackage{hyperref}
\usepackage[ruled,commentsnumbered]{algorithm2e}

\begin{document}

\maketitle

\begin{abstract}
    We present a model for cell growth, division and packing under soft constraints that arise from the deformability of the cells as well as of a membrane that encloses them. Our treatment falls within the framework of diffuse interface methods, under which each cell is represented by a scalar phase field and the zero level set of the phase field represents the cell membrane. One crucial element in the treatment is the definition of a free energy density function that penalizes cell overlap, thus giving rise to a simple model of cell-cell contact. In order to properly represent cell packing and the associated free energy, we include a simplified representation of the anisotropic mechanical response of the underlying cytoskeleton and cell membrane through appropriate penalization of the cell shape change. Numerical examples are presented to demonstrate the evolution of multi-cell clusters, and the total free energy of the clusters as a consequence of growth, division and packing. 
\end{abstract}

\section{Introduction}
Formation of multi-cell aggregates is a foundational process in the evolution of multicellular organisms. Beginning with a single cell or a small cluster, the growth of aggregates is driven by cell division, differentiation, migration and cell-cell interactions. Understanding the processes underlying the formation of these aggregates is central to many phenomena in cellular biology and physiology, including embryogenesis, regeneration, wound healing, tissue engineering, growth and metastasis in cancerous tumors, etc. As in most areas of cellular biology, a large body of work has focused on understanding the signalling pathways that control the evolution of multi-cell clusters and it is widely accepted that these pathways are triggered by the chemical environment and mechanical interactions (intra-cell, cell-cell, aggregate-matrix, external stimuli). However, the understanding of the spatial and temporal variations and effects of the chemo-mechanical environment in these cell aggregates is at a nascent stage and is increasingly leaning on computational models. 

Early work on modeling growth and interactions in cell clusters include lattice models \cite{GOEL1970423, GOEL1978103, Mochizuk1998} that treat cells as lattice sites in a square or hexagonal lattice and evolve multi-cell configurations through free energy minimizing cell pair exchanges. These highly reduced order representations have provided significant insights into the effect of cell-cell interactions on the evolution of cell aggregates. However, the cell-cell exchange processes assumed in these models are not universally observed in real cell aggregates. Such treatments also limit the incorporation of sub-cellular growth dynamics. Improvement of the lattice models in the form of sub-cellular lattice models using high-Q Potts models have delivered better geometric representation of cell structure \cite{Glazier1992, Glazier1993}. In sub-lattice models, the cells are represented by a cluster of lattice sites rather than a single lattice site, and cell migration is achieved by switching parent cell identity of the lattice sites at the boundary. A single such switch allows the  
cell boundary of one cell to advance by one lattice length into the neighbouring cell. These methods have allowed for a finer representation of the cell geometry, but result in unrealistic jagged cell boundaries and complex cell shapes that are not simply connected (e.g. cells within cells). 

The drawback of the lattice models with respect to a lack of representation of cell geometry or a jagged representation of the cell boundaries was partially addressed with the development of cell-centric/center-dynamics models \cite{HONDA1978523, HONDA1983191, GRANER1993455, Mosaffa2015} and vertex dynamics models \cite{HONDA1983191, Honda1986, FLETCHER20142291, Silvanus2017, Munoz2017}. The center dynamics models approximate cell shapes as polygons generated though Voronoi tesselation of a collection of forming points, and the evolution of cell boundaries is achieved through a free energy minimizing movement of the forming points. A major drawback of this method is the restriction of the boundary to a polygonal shape dictated by the underlying tessellation. This restriction was addressed in vertex dynamics models that allowed for the cells to be represented as general polygons defined by the connectivity of the vertices. This connectivity evolves dynamically and is driven by the free energy minimizing pair wise movement of vertices that conserve the cell volume. 

As detailed in the review paper by Brodland \cite{Brodland2004}, the lattice models, cell-centric models and the vertex dynamics models have successfully modeled a wide range of cell-cell interactions and cell aggregates phenomena. However, these models have no representation or a very simplified representation of cell geometries, cell-cell and cell-substrate interactions, cytoskeletal remodeling, cytoplasmic viscosity, etc. This greatly limits the ability to model realistic, smooth and anisotropic cell shape evolution, the mechanics of cell surface evolution due to cell-cell contact, the ability to control cell volumes and to ensure proper dissipative dynamics. Modeling and understanding these processes is central to characterizing the process of growth and evolution of multi-cell aggregates that we refer to as a problem of soft packing.  

In this manuscript, we present a finite element method based phase field representation of cells and the resulting soft packing dynamics of cell aggregates. The phase field method is a popular numerical technique for simulating diffuse interface kinetics at the mesoscale and has been widely used to model evolving interface problems like crystal growth, solidification and phase transformations in alloys. In this method, the evolution of a species concentration and/or phase is modelled using a set of conserved or non-conserved order parameters. The evolution of the order parameters and the corresponding interface kinetics are modeled as a system of parabolic partial differential equations, which are referred to as the Cahn-Hilliard formulation (for conserved order parameters) \cite{Cahn1958} and Allen-Cahn formulation (for non-conserved order parameters)\cite{Cahn1979}. The phase field representation of the cell geometries and the soft packing of multi-cell aggregates presented here allows for an improved representation of realistic cell shapes and a high fidelity representation of the growth, division, and mechanical compaction processes intrinsic to the formation of multi-cell aggregates. Migration also can be treated in this setting, although it is beyond the scope of this communication. An earlier attempt at modeling multi-cell aggregates using a phase field representation is outlined by Nonomura \cite{Nonomura2012} using an Allen-Cahn representation of the cells; i.e. a non-conserved order parameter treatment. In contrast, the formulation presented in this paper treats cell mass as a conserved quantity and models the cells using a Cahn-Hilliard representation. Furthermore, our treatment considers the mechanics of soft packing and allows for the necessary anisotropic shape evolution of cells. 

In Section \ref{sec:nonmechformulation}, we present the phase field formulation, its numerical implementation and simulations of cell growth and division. This is followed by the modeling of mechanics of soft packing and simulations of soft packing in Section \ref{sec:mechformulation}. Closing remarks appear in Section \ref{sec:concl}

\section{A phase field formulation for cell growth, division and contact}
\label{sec:nonmechformulation}
Our formulation of the problem rests on a phase field representation with as many scalar fields as cells. The treatment is centered on the definition of a free energy density, as a function of the scalar fields. In the non-mechanical version of the problem, terms are constructed to model cell membranes by phase segregation of cell interiors from the extra-cellular matrix, and contact inhibition, or intercellular repulsion, by penalizing overlapping scalar fields. We present the variational treatment, the mechanisms that model cell division, numerical aspects, and an illustrative numerical example.

\subsection{The diffuse interface model}
% Phase field formulation (without mechanics of shape change).
Let $\Omega\in \mathbb{R}^2$ with a smooth boundary $\partial \Omega$. Scalar fields $c_k, \;k = 1,\dots,N$ with $c_k \in [0,1]$ serve to delineate the interior and exterior of the cell numbered $k$. Here, the interior of cell $k$ is $\omega_k \subset \Omega$, where $\omega_k = \{ \bX \in \Omega\vert c_k(\bX) =1\}$. The exterior is $\Omega\backslash\omega_k$. The free energy density function is built up beginning with the following form: 
\begin{equation}
\psi_1(c_k) = \alpha c_k^2(c_k-1)^2 + \frac{\kappa}{2}\vert\grad c_k\vert^2
\label{equ:psi1}
\end{equation}
where the double-well term, $f(c_k) = \alpha c_k^2(c_k-1)^2$, enforces segregation into $\omega_k$ and $\Omega\backslash\omega_k$. In Equation \eqref{equ:psi1}, the second term enforces a diffuse cell-matrix interface (the cell membrane) of finite thickness, where $\kappa$ controls the interface thickness, and thereby the interfacial energy. For $N$ cells in $\Omega$, the above free energy density needs to be extended to model contact by adding a cell-cell repulsion term. The total free energy of the multi-cell aggregate is a functional $\Pi[\bc]$, defined as
\begin{align}
\Pi[\bc] &:= \int\limits_\Omega \psi(\bc,\nabla c) ~\text{d}V\nonumber\\
&=\int\limits_\Omega \left(\sum_{k=1}^{N} f(c_k) + \sum_{k=1}^{N} \frac{\kappa}{2}\vert\grad c_k\vert^2 + \sum_{l\ne k}\sum_{k=1}^{N}\lambda c_k^2 c_l^2 \right) ~\text{d}V.
\label{equ:energy}
\end{align}
Here, $\bc = \{c_1,\dots,c_N\}$, and $\lambda$ is a penalty coefficient that enforces repulsion between any two cells $k,l$ thus modelling cell contact.

Taking the variational derivative with respect to $c_k$ in Equation \eqref{equ:energy} yields
\begin{align}
\delta \Pi_k[\bc;w] =  & \left.\frac{\text{d}}{\text{d}\epsilon} \int\limits_{\Omega} \sum_{k=1}^{N} \left(f(c_k+\epsilon w) + \frac{\kappa}{2}\vert\grad (c_k+\epsilon w)\vert^2+ \sum_{l\ne k}\lambda (c_k+\epsilon w)^2 c_l^2\right)  ~\text{d}V \right|_{\epsilon=0} \nonumber\\
=&\int\limits_{\Omega} w \left( f^\prime(c_k) -  \kappa \Delta  c_k  + \sum_{l\ne k}2\lambda  c_k c_l^2 \right) ~dV + \int\limits_{\partial \Omega}   w \kappa \grad c_k \cdot \bn   ~dS
\end{align} 
where $\bn$ is the unit outward normal vector to $\partial \Omega$. The chemical potential of the $k^\text{th}$ cell is identified as,
\begin{equation}
\mu_k  = f^\prime(c_k) -  \kappa \Delta c_k + \sum_{l\ne k}2\lambda  c_k c_l^2
\label{equ:chemo_potential}
\end{equation}
At equilibrium, $\delta_k \Pi[\bc;w] =0$ for the $k^\text{th}$ cell, yielding $\mu_k = 0$ in $\Omega$, and $\kappa \grad c_k \cdot \bn = 0$ on $\partial \Omega$. \footnote{In some simulations, a buffer zone, $\Omega^'$, is needed around the simulation domain to inhibit unrealistic cell shapes resulting from the enforcement of $\kappa \grad c_k \cdot \bn = 0$ on $\partial \Omega$. In the buffer zone, an additional term of the form $\sum_{k=1}^{N}\lambda c_k^2$ is added to the free energy density to penalize the movement of any cells from the active simulation domain ($\Omega$) to the buffer zone ($\Omega^'$).} 

The following parabolic partial differential equation, popularly known as the Cahn-Hilliard equation \cite{Cahn1958}, imposes the conserved dynamics that governs the delineation and growth of the $N-$cell agglomerate, and of repulsion between cell pairs:
\begin{equation}
\frac{\partial c_k}{\partial t} = -~\grad \cdot (-M\grad \mu_k) + s_k
\label{equ:dynamic}
\end{equation}
where the source term $s_k$ has been introduced, and $M$ is the mobility, assumed to be constant. The dynamics of the multi-cell soft packing problem is governed by Equation \eqref{equ:dynamic} with the thermodynamics prescribed by Equation \eqref{equ:chemo_potential} and boundary conditions $\kappa \grad c_k \cdot \bn = 0$, $c_k=0$ on $\partial\Omega$ for $k = 1,\dots N$.

\subsection{Numerical implementation}
% Discuss the time step choice, and any other implementation issues, especially to ensure stability.
\label{sec:chemo_numerical}
Time discretization is carried out by the implicit, backward Euler method. Time instants are indexed by superscripts in the following development, and the time step is denoted by $\Delta t = t^{n+1}-t^n$. Starting with the initial conditions $\{c_k^0,\mu_k^0\}$, and given the solution $\{c_k^n,\mu_k^n\}$ at time $t^n$, the time-discrete versions of Equations \eqref{equ:dynamic} and \eqref{equ:chemo_potential} are,
\begin{align}
c^{n+1}_{k} &= c^{n}_{k} + \Delta t (M~\grad \cdot (\grad \mu^{n+1}_{k})+s_{k})\nonumber\\
\text{where}\quad\mu^{n+1}_{k} &= {f^\prime}^ {n+1}(c_k) -  \kappa \Delta c^{n+1}_{k} + \sum_{l\ne k}2\lambda  c_k^{n+1} {c_l^{n+1}}^2
\label{equ:time_discretization}
\end{align}

The weak form of the problem is: Find $c^{n+1}_k \in \mathscr{S} = \{c\in \mathscr{H}^1(\Omega)\vert c = 0\;\text{and}\; \nabla c \cdot\bn = 0 \;\text{on}\; \partial\Omega\}$ such that for any arbitrary variation $w \in \mathscr{V} = \{w\in \mathscr{H}^1(\Omega)\vert w = 0\;\text{and}\; \nabla w \cdot\bn = 0 \;\text{on}\; \partial\Omega\}$ on $c_k$, the following residual equations  are satisfied:
\begin{align}
\int\limits_{\Omega}   w c_k^{n+1} ~\text{d}V
&= \int\limits_{\Omega}   (w  c_k^{n} - \grad w \cdot \Delta t M \grad \mu_k^{n+1} + w\Delta t s_k ~\text{d}V\nonumber\\
\int\limits_{\Omega}   w  \mu_{k}^{n+1}  ~\text{d}V 
&=\int\limits_{\Omega}  (w  {f^\prime}^{n+1}(c_k) + \grad w \cdot \kappa \grad c_k^{n+1} ~dV + \int\limits_\Omega w \sum_{l\ne k}2\lambda  c_k^{n+1} {c_l^{n+1}}^2)  ~\text{d}V
\label{equ:weak}
\end{align}
Spatial discretization is implemented in a standard finite element framework and uses bilinear quadrilateral elements leading to standard matrix-vector forms of the equations in \eqref{equ:weak}.

\subsection{The model for cell division}
The total mass of cell $k$ at time $t^n$ is $m_k^{n} = \int_\Omega c_k^{n} ~\text{d}V$. When $m_k^n -2m_k^0 < \varepsilon_1$, where $\varepsilon_1 > 0$ is a tolerance, cell division is implemented by Algorithm \ref{Algo:division} below: A new scalar field is introduced by incrementing the number of cells $N\mapsto N+1$, and initializing a new scalar field $c_{N+1}$. For an elliptical cell dividing along its minor principal axis, $\omega_{N+1}^{n+1}$ is defined such that $\omega_k^{n}$ is bisected into $\omega_k^{n+1}$ and $\omega_{N+1}^{n+1}$, ensuring $\text{meas}(\omega_{k}^{n+1}) = 0.5\text{meas}(\omega_k^{n})$ and $\text{meas}(\omega_{N+1}^{n+1}) = 0.5\text{meas}(\omega_k^{n})$, where $\text{meas}(\omega_k)$ is the measure of $\omega_k$.
The regions $\omega_k^{n+1}$ and $\omega_{N+1}^{n+1}$ are thus determined by the divisions of elliptical cells along their minor principal axes. This strategy of bisecting an elliptical cell along its minor axes is motivated by observations of cell division in biological cells. However, recognizing that the shape of $\omega_k^n$ will, in general deviate from an ellipse, we define the division axis to lie along the axis of the major principal moment of inertia through the center of mass. Note that, for an ellipse, the minor principal axis, and the major principal moment of inertia axis through the center of mass coincide. At time $t^{n+1}$, a new interface forms between $\omega_k^{n+1}$ and $\omega_{N+1}^{n+1}$ following a division of the $k^\text{th}$ cell at time $t^n$ that incremented the number of cells $N\mapsto N+1$. This new interface is along the major principal axis of the moment of inertia tensor through the center of mass of $\omega^n_k$; that is, of the $k^\text{th}$ cell's interior at time $t^n$. 

\begin{algorithm}[h]
$m_k^{n} = \int_\Omega c_k^{n} ~dV$\;
\If{$m_k^n -2m_k^0 < \varepsilon_1$}{
    $N\mapsto N+1$\;
    $\text{meas}(\omega_{k}^{n+1}) = 0.5\text{meas}(\omega_k^{n})$\;  $\text{meas}(\omega_{N+1}^{n+1}) = 0.5\text{meas}(\omega_k^{n})$\;
    %$m(\omega_k^{n+1}) = 0.5m(\omega_k^{n})$\;
    %$m(\omega^{n+1}_N) = m(\omega^{n+1}_k)$\;
}
\caption{Cell division mechanism}
\label{Algo:division}
\end{algorithm}

\subsection{Adaptive time stepping}
A uniform time step, $\overline{\Delta t}$, is chosen such that it ensures the stability and convergence of the Cahn-Hilliard dynamics. However, when cell divisions occur, the sharp variation of the composition field at the boundary of the two daughter cells necessitates an adaptive time step control to equilibrate the sharp composition fields over the new daughter cell boundaries. To address this transient instability, $\Delta t$ is decreased by a factor of $1.0\times 10^{-m}$ for a few time steps ($n_\text{div}$) in order to ensure convergence of the nonlinear system of equations. The time step is then sharply increased back to $\overline{\Delta t}$ until the next cell division process. Algorithm \ref{Algo:Timestep} details the implementation of this adaptive time step control in our code. Typical values used for the simulations presented in this paper are $n_{div}<5$ and $m<7$.

\begin{algorithm}[h]
$\Delta t=\overline{\Delta t}$\;
\For{$t\leftarrow 0$ \KwTo T ; $t^{n+1}=t^n+\Delta t$}{
    \tcc{T is total time of computation; $\Delta t$ is time step}
    \If{cell is about to divide}{
        $\Delta t=\overline{\Delta t}\times 1.0\times 10^{-m}$\;
        $a=0$\;
        \If{$\Delta t<\overline{\Delta t}$}{
            $a=a+1$\;
            \If{$a>n_{div}$}{
                $\Delta t=\overline{\Delta t}\times 1.0\times 10^{-m}\times(a-n_{div})^m$\;
            }
        }
    }
}
\caption{Adaptive time step control\label{Algo:Timestep}}
\end{algorithm}

\subsection{Adaptive mass source}
\label{sec:source}
The growth of cells is controlled by the source term, $s_k$, which can be determined from experimentally observed cell doubling time estimates. In this work, $s_k$ is a function of the mass ratio $\nu_k^n = m_k^n/m_k^0$: The default value of $s_k=\overline{s}$, the average growth rate. This growth continues until the cell has doubled in mass and then the division process separates the cell into two daughter cells. The two daughter cells then continue to grow at the rate $\overline{s}$. However, the total number of cells in the limit of optimal soft packing is given by $N_\text{max}=\text{meas}(\Omega)/m_k^0$. If either the number of cells approaches $N_\text{max}$ or if no new scalar fields are available to initialize new daughter cells leading to $\nu_k^n > 1.1$, we turn off the source, $s_k^{n+1} = 0$, so that cell $k$ no longer grows. This adaptive mass source control appears in Algorithm \ref{Algo:Source}.

\begin{algorithm}[h]
\If{$\bX \in \omega^n_k$; in cell $k$}{
    $s^n_k = \overline{s}$\;
    $\nu^n_k = m^n_k/m^0_k$\;
    \tcc{ratio of current cell mass to its initial mass}
    \If{$N \ge N_\text{max}$; new phase field is not available}{
        \If{$\nu^n_k > 1.1$}{
            $s^n_k = 0.0$\;
        }
    }
    
}
\caption{Adaptive mass source\label{Algo:Source}}
\end{algorithm}

\subsection{Code framework}
The two-dimensional, cell growth and soft packing formulation presented here has been implemented in the \texttt{C++} based \texttt{deal.II} open source parallel finite element library \cite{BangerthHartmannKanschat2007}. We use the \texttt{SuperLU} direct solver \cite{li05} to solve the system of linear equations obtained from the linearization of Equation \eqref{equ:weak}. The linearization itself is obtained using the \texttt{Sacado} algorithmic differentiation library of the open source \texttt{Trilinos} project \cite{Trilinos2003}.  

%The code for all numerical examples in this paper is available at \href{https://gitlab.com/compPhysCode/softPacking}{\texttt{gitlab.com/compPhysCode/softPacking}}. Post-processing was carried out in the \texttt{VisIt} open-source software.

\subsection{An illustrative example for the progression of cell growth, division and conntact}
Our computations begin with a single cell, $\omega_1^0$, of circular shape at the center of an elliptical ``embryo'' $\Omega$. The cell grows under the action of the mass source. The parameters used in this numerical example appear in Table \ref{tab:parameter}. The simulation for $N_\text{max} = 12$ is shown in Figure \ref{fig:celldivision}. Each division axis is determined by the major principal direction of the moment of inertia tensor through the center of mass, which distributes mass evenly in each daughter cell. This can be observed from the progression between Figures \ref{fig:onecell} and \ref{fig:twelvecell}. Attention is also drawn to the delineation of daughter cells following each division, and of non-sibling cells from each other due to the repulsion terms in Equation \eqref{equ:energy}.  Figure \ref{fig:energy} tracks the total free energy of the system. Note that the total free energy increases with time due to the mass supply, and that transient fluctuations occur at each cell division due to the formation of a sharp boundary between the daughter cells and the transiently stronger repulsion between them. The initial condition for the phase field has a small random perturbation from a mean value to help drive the Cahn-Hilliard dynamics. This perturbation results in a tilt in the major principal direction of the moment of inertia tensor, and asymmetric division of the cells ($\sim 0.05$ radians off the symmetric axes).
\begin{table}[h!]
	\caption{Numerical values of parameters}
	\begin{center}
		\begin{tabular}{|c|c|c|c|c|c|c|}
			\hline
			Parameters & $\alpha$ & $\kappa$ & $\lambda$ & $M$ & $s$ & $\overline{\Delta t}$\\
			\hline
			Value & $4$ & $1.0\mathrm{e}{-3}$ & $100$ & $1$ & $5.0\mathrm{e}2$ & $2.0\mathrm{e}{-4}$\\
			\hline
		\end{tabular}
	\end{center}
	\label{tab:parameter}
\end{table}

\begin{figure}[h!]
	\centering
	\begin{subfigure}[h!]{0.4\textwidth}
		\includegraphics[width=\textwidth]{figs/onecell_2}
		\caption{Initial single circular cell}
		\label{fig:onecell}
	\end{subfigure}
	\begin{subfigure}[h!]{0.4\textwidth}
		\includegraphics[width=\textwidth]{figs/fourcell_2}
		\caption{Progression to four cells}
		\label{fig:fourcell}
	\end{subfigure}
	\begin{subfigure}[h!]{0.4\textwidth}
		\includegraphics[width=\textwidth]{figs/eightcell_2}
		\caption{Progression to eight cells}
		\label{fig:eightcell}
	\end{subfigure}
	\begin{subfigure}[h!]{0.4\textwidth}
		\includegraphics[width=\textwidth]{figs/twelvecell_2}
		\caption{Progression to twelve cells.}
		\label{fig:twelvecell}
	\end{subfigure}
	\caption{A demonstration of the progression of cell division from one cell into twelve cells. Cell interiors are shown in red and the cell membrane in cyan-yellow.}
	\label{fig:celldivision}
\end{figure}

\begin{figure}[h!]
\centering
\includegraphics[scale=0.5]{figs/energy}
\caption{Evolution of total free energy with time (normalized). Each spike in energy corresponds to transient repulsion between newly formed daughter cells following a cell division event. In some cases, several cell divisions happen at the same time, leading to larger spikes.}
\label{fig:energy}
\end{figure}

\iffalse
\begin{figure}[h!]
\centering
\includegraphics[scale=0.2]{figs/mesh}
\caption{A demonstration of the progression of cell division. Use this as an example for the other figures. Replace this with a single collage of figure showing the progression of 1-12 cells. Show no legends, bounding box, etc.}
\label{fig:celldivision1}
\end{figure}
\fi

\section{Mechanics of soft packing, driven by cell shape changes}
\label{sec:mechformulation}
We now present an extension of the formulation to an elementary mechanical model that associates energy to global shape changes. The model works by penalizing departures of the principal moments of inertia from their initial values, which are established when a specific cell is created. A more complete treatment of mechanics in the cell growth, division and soft packing problem would consider nonlinearly elastic interactions between cells. However, as we demonstrate, the simple shape change-based model does capture the essential mechanics of soft packing. We include a study of the reduced moduli that govern shape change via a parametric study, before studying a sequence of cell growth, divisions and soft packing with the added mechanical contributions.

\subsection{An extension of the free energy density function to incorporate the mechanics of cell shape change}
% Phase field formulation extended to shape change via principal moments of inertia.
Consider the addition of the following term to the free energy density \eqref{equ:energy} in order to account for the principal moments of inertia of a single cell: 
\begin{equation}
\Pi_{\text{MI}} = \sum_{i=1}^{\text{dim}}\delta^i ({I^i} ^\text{ref}-I^i)^2
\label{equ:MI}
\end{equation}
Here,  $\delta^i$ is a mechanical modulus penalizing variations in the $i^\text{th}$ principal moment of inertia, $I^i$ from its  reference state, ${I^i}^\text{ref}$, established at the birth of this cell. With this addition that accounts for changes in shape, the total free energy function takes on the form:
\begin{align}
\Pi[\bc,\nabla\bc] &:= \int\limits_{\Omega} \left(\sum_{k=1}^{N} f(c_k) + \sum_{k=1}^{N} \frac{\kappa}{2}\vert\grad c_k\vert^2 + \sum_{l\ne k}\sum_{k=1}^{N}\lambda c_k^2 c_l^2 \right) ~\text{d}V\nonumber\\
&\phantom{+}+ \sum_{k=1}^{N}\sum_{i=1}^{\text{dim}}\delta_k^i ({I_k^i} ^\text{ref}-I_k^i)^2
\end{align}
The moment of inertia tensor through the center of mass, $\bI$, is expressed in terms of its Cartesian components $I_{ij}$, with the phase field playing the role of the mass density in the traditional definition of this quantity:
\begin{equation}
\bI[c] = 
\begin{bmatrix}
I_{11}[c] & I_{12}[c] \\
I_{21}[c] & I_{22}[c]
\end{bmatrix}
=
\begin{bmatrix}
\int c \bar{X}_2^2 ~\text{d}V   & -\int c\bar{X}_1 \bar{X}_2 ~\text{d}V\\
-\int c\bar{X}_1 \bar{X}_2 ~\text{d}V & \int c\bar{X}_1^2 ~\text{d}V
\end{bmatrix}
\label{equ:MomentOfInertia}
\end{equation}
where, with the center of mass
\begin{equation}
    X^c_i = \frac{\int_\Omega c X_i ~\text{d}V}{\int_\Omega c ~\text{d}V},
    \label{equ:masscenter}
\end{equation} the Cartesian coordinates relative to the center of mass are $\bar{X}_{i} = X_{i}-X^c_{i}$. Then, the principal moments of inertia through the center of mass, $I^i$, can be determined from the moment of inertia tensor in Equation \eqref{equ:MomentOfInertia} by an eigen decomposition governed by the Cayley-Hamilton Theorem:
\begin{equation}
    I_k^2 - I_k\text{tr}\bI_k + \text{det}\bI_k = 0
    \label{equ:CHThm}
\end{equation}
where $I_k$ denotes a principal value of the moment of inertia tensor, relative to the center of mass, of the $k^\text{th}$ cell, $\bI_k$.

The variational machinery applied to computing the chemical potential now must account for the added shape-dependent, moment of inertia terms:
\begin{align}
\delta \Pi_k[\bc;w]=  &\frac{\text{d}}{\text{d}\epsilon} \int\limits_{\Omega} \sum_{k=1}^{N}\left( f(c_k+\epsilon w) + \frac{\kappa}{2}\vert\grad (c_k+\epsilon w)\vert^2 + \sum_{l\ne k}\lambda (c_k+\epsilon w)^2 c_l^2 \right) ~\text{d}V \nonumber\\
&+ \frac{\text{d}}{\text{d}\epsilon}\left. \sum_{k=1}^{N}\sum_{i=1}^{\text{dim}}\delta_k^i \left({I_k^i} ^\text{ref}-I^i_k[c_k+\epsilon w]\right)^2 \right\vert_{\epsilon=0} \nonumber\\
= &\sum_{k=1}^{N} \int\limits_{\Omega} w \left( {f^\prime}(c_k) -  \kappa \Delta  c_k  + \sum_{l\ne k}2\lambda  c_k c_l^2 \right) ~\text{d}V \nonumber \\
&- \sum_{k=1}^N\sum_{i=1}^{\text{dim}} 2\delta_k^i \left({I_k^i} ^\text{ref} - I_k^i[c_k]\right)\tilde{I}^i_k[c_k] \nonumber	\\
&+ \int\limits_{\partial \Omega}   w \kappa \grad c_k \cdot \bn   ~dS
\end{align}
where, the term $\tilde{I}^i_k$ arises from the variation of $I^i_k$, and is obtained from the Cayley-Hamilton Theorem to be:

\begin{equation}
    \tilde{I}^i_k = \frac{\int\limits_\Omega w\left(I^i_k\text{tr}\bar{\bI}-\bar{X}_1^2\int\limits_\Omega c\,\bar{X}_2^2\,\text{d}V -\bar{X}_2^2\int\limits_\Omega c\,\bar{X}_1^2\,\text{d}V+2\bar{X}_1\bar{X}_2\int\limits_\Omega c\,\bar{X}_1\bar{X}_2\text{d}V\right)\text{d}V}{2 I^i_k - \text{tr}\bI},
    \label{equ:CayleyHamilton}
\end{equation}
with the tensor 
\begin{equation}
    \bar{\bI} = 
    \begin{bmatrix}
\bar{X}_2^2   & -\bar{X}_1 \bar{X}_2\\
-\bar{X}_1 \bar{X}_2 & \bar{X}_1^2
\end{bmatrix}.
\label{equ:barI}
\end{equation}

Extending Equation \eqref{equ:chemo_potential}, the chemical potential is now defined as
\begin{equation}
\mu_k  = f^\prime(c_k) -  \kappa \Delta c_k + \sum_{l\ne k}2\lambda  c_k c_l^2 - \sum_{i=1}^{\text{dim}} 2\delta_k^i \left({I_k^i} ^\text{ref} - I_k^i\right)\hat{I}^i_k,
\label{equ:chempotmech}
\end{equation}
where $\hat{I}^i_k$ is
\begin{equation}
    \hat{I}^i_k = \frac{I^i_k\,\text{tr}\bar{\bI} - \bar{X}_1^2\,\int\limits_\Omega c\,\bar{X}_2^2\,\text{d}V - \bar{X}_2^2\,\int\limits_\Omega c\,\bar{X}_1^2\,\text{d}V+ 2\bar{X}_1\bar{X}_2\,\int\limits_\Omega c\,\bar{X}_1\bar{X}_2\,\text{d}V}{2 I^i_k - \text{tr}\bI},
    \label{equ:Ihat}
\end{equation}
thus introducing a simple mechanical model that penalizes shape changes of the cell. When combined with the governing parabolic partial differential equation in conservation form \eqref{equ:dynamic}, and the boundary conditions $\kappa \grad c_k \cdot \bn = 0$, $c_k=0$ on $\partial\Omega$ for $k = 1,\dots N$, we have a description for multi-cell growth, division and soft packing with the penalization of shape change. 

\subsection{Numerical implementation of the extended model}
% Any additional issues: choice of shape change stiffness parameters, etc.
The time discretized dynamics are now written as an explicit-implicit scheme. Given the initial conditions $\{c_k^0,\mu_k^0\}$ and the solution $\{c_k^n,\mu_k^n\}$, the time-discrete versions of Equations \eqref{equ:dynamic} and \eqref{equ:chempotmech} are,
\begin{align}
c^{n+1}_{k} &= c^{n}_{k} + \Delta t (M~\grad \cdot (\grad \mu^{n+1}_{k})+s_{k})\nonumber\\
\mu_k^{n+1}  &= f_{,c_k^{n+1}} -  \kappa \Delta c_k^{n+1} + \sum_{l\ne k}2\lambda  c_k^{n+1} {c_l^{n+1}}^2 - \sum_{i=1}^{\text{dim}} 2\delta_k^i \left({I_k^i} ^\text{ref} - I_k^i\right)\hat{I}^i_k.
\label{equ:anisotropy_time_discretization}
\end{align}
Note that the mechanical terms exerting control over the shape do not carry time instant superscripts in Equation \eqref{equ:anisotropy_time_discretization} for the sake of brevity. We have experimented with evaluating these terms at $t^{n+1}$ in a fully implicit method, as well as at $t^n$, in an explicit-implicit implementation. The complicated functional evaluations in Equations \eqref{equ:CayleyHamilton}-\eqref{equ:Ihat} make the fully implicit method notably more expensive than the explicit-implicit method, which has been used in the numerical examples of Section \ref{sec:anisotropy_study}.

The weak form follows: Find $c^{n+1}_k \in \mathscr{S} = \{c\in \mathscr{H}^1(\Omega)\vert c = 0\;\text{and}\; \nabla c \cdot\bn = 0 \;\text{on}\; \partial\Omega\}$ such that for any arbitrary variation $w \in \mathscr{V} = \{w\in \mathscr{H}^1(\Omega)\vert w = 0\;\text{and}\; \nabla w \cdot\bn = 0 \;\text{on}\; \partial\Omega\}$ on $c_k$ the following residual equations  are satisfied:

\begin{align}
\int\limits_{\Omega}   w c_k^{n+1} ~dV
=&\int\limits_{\Omega}   w  c_k^{n} - \grad w \cdot \Delta t M \grad \mu_k^{n+1} + ws_k ~dV\nonumber\\
\int\limits_{\Omega}   w  \mu_{k}^{n+1}  ~dV 
=&\int\limits_{\Omega}  w  f_{,c_k}^{n} + \grad w \cdot \kappa \grad c_k^{n} ~dV + \int\limits_\Omega w \sum_{k\ne l}2\lambda  c_k c_l^2  ~dV \nonumber\\
&- \sum_{i=1}^\text{dim} 2\delta_k^i \left({I_k^i} ^\text{ref}- I_k^i\right)\int\limits_\Omega w\, \hat{I}^i_k ~dV
\label{equ:anisotropy_weak} 
\end{align}

\subsection{Parametric study of the mechanical moduli}
In our numerical experiments with the above model for mechanical control of shape change in packing, we have found that the effective values of the mechanical moduli $\delta^i_k$ that impose control on cell shape fall within the range $\delta^i_k \in [1\times 10^4, 2\times 10^5]$. Thus, with $\delta^1_k \to 0$ and $\delta^2_k$ in the above range, the model produces elliptical single cells of increasing aspect ratio with the major axis corresponding to the smaller principal moment of inertia $I^2_k < I^1_k$. This parametric study appears in Figure \ref{fig:penalization}. Much larger values  $\delta^i_k > 2\times 10^5$ make the problem stiff, and convergence of the nonlinear solver becomes difficult. Other aspects of the numerical implentation remain the same as in Sections \ref{sec:chemo_numerical}-\ref{sec:source}. 

%\subsection{Extension of the code framework}
%The finite element uses the same libraries and methods discussed in Section \ref{sec:chemo_numerical}. In %addition, the numerical computation of the eigen decomposition relies on the function \texttt{Xsyevx} from the %\texttt{LAPACK} library.

\subsection{Numerical studies of soft packing of cells with penalization of shape changes}
% Results showing single cell changing shape due to anisotropic penalization, and progression of multi-cell clusters with 1-4 cells.
\label{sec:anisotropy_study}
In order to illustrate the control of cell shape  by the mechanical moduli $\delta^i_k$, we set $\delta^1_k \to 0$ and $\delta^i_k \in [1\times 10^4, 2\times 10^5]$. The results appear in Figure \ref{fig:penalization} for a single cell. The modulus $\delta^2_1$ takes on values $5\times 10^4$, $1\times 10^5$, and $2\times 10^5$ in Figures \ref{fig:delta50000}, \ref{fig:delta100000} and \ref{fig:delta200000}, respectively. It can be seen from these three figures that the larger mechanical modulus, $\delta^2_1$ constrains growth along the minor principal axis of the elliptical cell that forms. As $\delta^2_1$ becomes larger, it tends to produce an oblate shape, as the final equilibrium state in Figure \ref{fig:delta200000}. In comparison, at the lower value of $\delta^2_1 = 5\times 10^4$, this modulus does not much affect the ellipticity of cell shape, as seen in Figure \ref{fig:delta50000}. 

The progression of cell division from one single cell into twelve daughter cells is demonstrated in Figure \ref{fig:anisotropy_division}. Note the tight packing and shape changes attained by the cells. We also draw attention to the differences in cell shapes between Figures \ref{fig:celldivision} and \ref{fig:anisotropy_division}. This difference is especially notable at the 12-cell stage, and is  due to the added penalization of cell shape change already demonstrated in Figure \ref{fig:penalization}. We expect, also, that the shapes in Figure \ref{fig:anisotropy_division} are more physically accurate because they account for shape change, albeit by a simple mechanical model. 

The accompanying total free energy evolution is shown in Figure \ref{fig:energy_anisotropy}. Like in Figure \ref{fig:energy}, transient fluctuations occur at each cell division due to the formation of a sharp boundary between the daughter cells and the transiently stronger repulsion between the daughter cells. However, in this case, the transient fluctuations are more prominent and spread out due to the rapid changes in daughter cell shapes that themselves result from the elastic repulsion following division. Compared to Figure \ref{fig:energy}, the free energy values in Figure \ref{fig:energy_anisotropy} are much higher. This is due to the additional penalization in the form of the mechanics term whose relative magnitude has been made higher than the regular Cahn-Hilliard and overlap penalization terms. As observed before, the height of a spike and its width on the time axis increase with the number of cell divisions occurring in that time interval and the mass source causes the gradual increase in the free energy over time.

\begin{figure}[h!]
	\centering
	\begin{subfigure}[h!]{0.45\textwidth}
		\includegraphics[width=\textwidth]{figs/circley50000_2}
		\caption{$\delta_2 = 50000$}
		\label{fig:delta50000}
	\end{subfigure}
	\begin{subfigure}[h!]{0.45\textwidth}
		\includegraphics[width=\textwidth]{figs/circley100000_2}
		\caption{$\delta_2 = 100000$}
		\label{fig:delta100000}
	\end{subfigure}
	\begin{subfigure}[h!]{0.45\textwidth}
		\includegraphics[width=\textwidth]{figs/circley200000_2}
		\caption{$\delta_2 = 200000$}
		\label{fig:delta200000}
	\end{subfigure}
	\caption{Fully developed single cell shape change due to anisotropic mechanical moduli $\delta^1_k \to 0$ and $\delta^2_k \in [1\times 10^4, 2\times 10^5]$.}
	\label{fig:penalization}
\end{figure}

\begin{figure}[h!]
	\centering
	\begin{subfigure}[h!]{0.45\textwidth}
		\includegraphics[width=\textwidth]{figs/anisotropy_onecell_2}
		\caption{one cell}
		\label{fig:anisotropy_onecell}
	\end{subfigure}
	\begin{subfigure}[h!]{0.45\textwidth}
		\includegraphics[width=\textwidth]{figs/anisotropy_fourcell_2}
		\caption{Four cells}
		\label{fig:anisotropy_fourcell}
	\end{subfigure}
	\begin{subfigure}[h!]{0.45\textwidth}
		\includegraphics[width=\textwidth]{figs/anisotropy_eightcell_2}
		\caption{Eight cells}
		\label{fig:anisotropy_eightcell}
	\end{subfigure}
	\begin{subfigure}[h!]{0.45\textwidth}
		\includegraphics[width=\textwidth]{figs/anisotropy_twelvecell_2}
		\caption{Twelve cells}
		\label{fig:anisotropy_twelvecell}
	\end{subfigure}
	\caption{Cell division with penalization of cell shape change. The shapes of the 12-cell cluster differ from the 12-cell cluster in Figure \ref{fig:celldivision} due to the additional effect of mechanics. Cell interiors are shown in red and the cell membrane in cyan-yellow.}
	\label{fig:anisotropy_division}
\end{figure}

\begin{figure}[h!]
\centering
\includegraphics[scale=0.5]{figs/energy_anisotropy}
\caption{Evolution of the total free energy with time (normalized). The increase in the free energy is due to the addition of mass, while the sharp spikes correspond to the repulsion between newly formed daughter cells, and penalization of their shape changes.}
\label{fig:energy_anisotropy}
\end{figure}

\section{Conclusion}
\label{sec:concl}
A phase field-based diffuse interface framework for modeling evolution of multi-cell aggregates has been presented in this paper. The model allows for high fidelity representation of smooth, anisotropic cell geometries, cell-cell contact evolution and the resulting mechanical compaction processes intrinsic to soft packing in multi-cell aggregates. The cells are represented as conserved scalar phase fields driven by the growth, division and compaction processes. The zero level set of each phase field identifies the cell membrane. 

The driving force is the minimization of a free energy function and the governing equations are variationally derived to yield a system of parabolic partial differential equations. The salient features of the formulation are: (i) Being a field formulation, no discreet interface evolving mechanisms like those employed in lattice, cell-centric and vertex dynamics models are needed. (ii) The time evolution occurs at realistic time scales controlled by the growth rate, or doubling time of the cells, and does not need the fine scale time stepping  and equilibrations needed by other discrete models. (iii) The cell boundary is represented by a continuously differentiable zero level set of the phase field and can thus represent any arbitrary cell shape without being limited to polygonal shapes or a jagged representation of the cell boundary. (iv) The mechanics of soft packing is captured by penalizing departures of the principal moments of inertia of each cell from their initial values. This is a very simplified representation of the anisotropic mechanical response of the underlying cytoskeleton and cell membrane.

The formulation presented in this paper does not capture some essential mechanisms like cell-cell adhesion, cell-matrix interactions and the mechanics of the cell cortex that can influence soft packing. However their inclusion is possible by adding appropriate gradient penalizations at the cell boundary to the free energy density function. These, and the incorporation of cell migration will be addressed in a future work, as will be the extension of the model to three dimensions. Some potential applications of this framework are in modeling the effects of mechanics and compaction in embryogenesis, soft packing in cancerous tumors and modeling of wound healing and tissue growth. 

\bibliographystyle{plain}
\bibliography{references}
\end{document}
