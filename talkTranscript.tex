\documentclass[a4paper,oneside,11pt]{report}
\usepackage{amsmath}
\begin{document}
	Slide 1
	\begin{itemize}
		\item Model for cell growth, cell division and packing 
		\item Under soft constraints
		\item Phase field modeling (Diffuse inteface methods)
		\item Using one of many computational methods: Finite element framework
		\item Membrane: Zero level set
	\end{itemize}
   Slide 2
   \begin{itemize}
   	\item Mathematical problem, packing of spheres
   	\item Fixed volume. Optimal arrangement
   	\item Inaccurate representation with right
   	\item Shows embryo of starfish and cell evolution in a fixed volume.
   	\item Challenge is to develop numerical methods to model alongside, incorporate the physics with it.
   \end{itemize}
   Slide 3
   \begin{itemize}
   	\item Peruse though the motivation slide 
   	\item Previous literature studies based on it, Our numerical framework and
   	\item Demonstrate certain examples
   \end{itemize}
   Slide 4
   \begin{itemize}
   	\item Figure shows light microscopy embryonic diagram of a sea urchin
   	\item Nucleus divides, as indicated by arrows.
   	\item Division proceeds evenly and then due to the fixed volume confinement it starts packing
   	\item it forms a morula after multiple anisotropic cell divisions. Motivation
   \end{itemize}
   Slide 5
   \begin{itemize}
   	\item Another problem which serves as our motivation is tumor growth
   	\item Dense stromal tissue prevents drug transport
   	\item Interactions between fibronectins, calcinogens, cancerous fibrogens cause this. Investigate
   \end{itemize}
   Slide 6
   \begin{itemize} 
    \item Previous literature analysed these issues (on lattice and off latice)and came up with cellular automata
    \item Collection of colored sets on structured grids which depend on rules set by neighbouring sets/cells
    \item Jagged boundaries, crude model
    \item Cannot capture cell interaction, cell division etc
    \end{itemize}
   Slide 7
   \begin{itemize}
   	\item Off latice addressed this problem of jagged boundaries and came up with vertex models
   	\item Though it represents collection of cells broadly
   	\item Cell shapes are primitive, minimally represents the original smooth shapes
   \end{itemize}
   Slide 8
   \begin{itemize}
   	\item Now we demonstrate our numerical framework to model these system
   	\item illustrates cell division progressively.
   	\item The cells are non overlapping and have anisotropy and smoothness
   	\item The ellipse is an user defined boundary. Blue indicates cell exterior.
   	\item Red indicates cell cytoplasm and the intermediate rainbow like transition from red to blue indicates cell membrane
   	\item our frame work is based on volume compaction
   \end{itemize}
   Slide 9
   \begin{itemize}
   	\item Two categories of equations
   	\item Solves 2nd order and 4th order pdes
   	\item Can model both conserved and non conserve fields
   \end{itemize}
   Slide 10
   \begin{itemize}
   	\item free energy density function which depends on order parameter and its spatial derivative
   	\item Scalar fields are considered as individual cells
   	\item Bulk energy term, diffuse interface membrane, $\kappa$ control thickness
   	\item Penalty term enforces cellular repulsion or non overlapping of scalar fields
   \end{itemize}
   Slide 11
   \begin{itemize}
   	\item we take a standard variation formulation
   	\item Resulting chemical concentration is what we are interested in (scalar field, order parameter)
   	\item Home grown code
   \end{itemize}
   Slide 12
   \begin{itemize}
   	\item Show video. Explain division from 2 to 4
   	\item Time scales are similar
   	\item Anisotropic growth, reiterate the color legend
   	\item Packing of soft deformable structures illustrated
   \end{itemize}
   Slide 13
   \begin{itemize}
   	\item Refresh what we said so far
   	\item Unlike other methods we do not have interface evolving mechanics
   \end{itemize}
    
\end{document}