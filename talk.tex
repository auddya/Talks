\documentclass[compress,red, 9pt, t]{beamer}
\mode<presentation>
%\usetheme{Singapore}
%\hypersetup{pdfpagemode=FullScreen}
\useoutertheme[subsection=false]{smoothbars}
 
\addtobeamertemplate{navigation symbols}{}{%
    \usebeamerfont{footline}%
    \usebeamercolor[fg]{footline}%
    \hspace{1em}%
    \insertframenumber/\inserttotalframenumber
}

% include packages
%\usepackage{subfigure}
\usepackage{multicol}
\usepackage{amsmath}
\usepackage{psfrag}
\usepackage{epsfig}
\usepackage{graphicx}
\usepackage[all,knot]{xy}
\xyoption{arc}
\usepackage{url}
\usepackage{multimedia}
\usepackage{hyperref}
\usepackage{setspace}
\setbeamersize{text margin left=4mm, text margin right=3mm}
%My Packages
\usepackage{stmaryrd}
\usepackage{mathrsfs}
%\usepackage[font=small,format=plain,labelfont=bf,up,textfont=it,up]{caption}
\usepackage{rotating}
\usepackage{textpos}
\usepackage{captdef}
\usepackage{caption}
\usepackage{subcaption}

\usepackage{rotating}
\newcommand{\sun}{\ensuremath{\odot}} % sun symbol is \sun
\let\vaccent=\v % rename builtin command \v{} to \vaccent{}
\renewcommand{\v}[1]{\ensuremath{\mathbf{#1}}} % for vectors
\newcommand{\gv}[1]{\ensuremath{\mbox{\boldmath$ #1 $}}} 
\newcommand{\grad}[1]{\gv{\nabla} #1}
%---------------------------------------------------------
%               Bold Face Math Characters:
%               All In Format: \B***** .
%---------------------------------------------------------
\def\BGamma{\mbox{\boldmath$\partial\Omega$}}
\def\BDelta{\mbox{\boldmath$\Delta$}}
\def\BTheta{\mbox{\boldmath$\Theta$}}
\def\BLambda{\mbox{\boldmath$\Lambda$}}
\def\BXi{\mbox{\boldmath$\Xi$}}
\def\BPi{\mbox{\boldmath$\Pi$}}
\def\BSigma{\mbox{\boldmath$\Sigma$}}
\def\BUpsilon{\mbox{\boldmath$\Upsilon$}}
\def\BPhi{\mbox{\boldmath$\Phi$}}
\def\BPsi{\mbox{\boldmath$\Psi$}}
\def\BOmega{\mbox{\boldmath$\Omega$}}
\def\Balpha{\mbox{\boldmath$\alpha$}}
\def\Bbeta{\mbox{\boldmath$\beta$}}
\def\Bgamma{\mbox{\boldmath$\gamma$}}
\def\Bdelta{\mbox{\boldmath$\delta$}}
\def\Bepsilon{\mbox{\boldmath$\epsilon$}}
\def\Bzeta{\mbox{\boldmath$\zeta$}}
\def\Beta{\mbox{\boldmath$\eta$}}
\def\Btheta{\mbox{\boldmath$\theta$}}
\def\Biota{\mbox{\boldmath$\iota$}}
\def\Bkappa{\mbox{\boldmath$\kappa$}}
\def\Blambda{\mbox{\boldmath$\lambda$}}
\def\Bmu{\mbox{\boldmath$\mu$}}
\def\Bnu{\mbox{\boldmath$\nu$}}
\def\Bxi{\mbox{\boldmath$\xi$}}
\def\Bpi{\mbox{\boldmath$\pi$}}
\def\Brho{\mbox{\boldmath$\rho$}}
\def\Bsigma{\mbox{\boldmath$\sigma$}}
\def\Btau{\mbox{\boldmath$\tau$}}
\def\Bupsilon{\mbox{\boldmath$\upsilon$}}
\def\Bphi{\mbox{\boldmath$\phi$}}
\def\Bchi{\mbox{\boldmath$\chi$}}
\def\Bpsi{\mbox{\boldmath$\psi$}}
\def\Bomega{\mbox{\boldmath$\omega$}}
\def\Bvarepsilon{\mbox{\boldmath$\varepsilon$}}
\def\Bvartheta{\mbox{\boldmath$\vartheta$}}
\def\Bvarpi{\mbox{\boldmath$\varpi$}}
\def\Bvarrho{\mbox{\boldmath$\varrho$}}
\def\Bvarsigma{\mbox{\boldmath$\varsigma$}}
\def\Bvarphi{\mbox{\boldmath$\varphi$}}
\def\bone{\mbox{\boldmath$1$}}
\def\bzero{\mbox{\boldmath$0$}}
%---------------------------------------------------------
%               Bold Face Math Italic:
%               All In Format: \b* .
%---------------------------------------------------------
\def\bA{\mbox{\boldmath$ A$}}
\def\bB{\mbox{\boldmath$ B$}}
\def\bC{\mbox{\boldmath$ C$}}
\def\bD{\mbox{\boldmath$ D$}}
\def\bE{\mbox{\boldmath$ E$}}
\def\bF{\mbox{\boldmath$ F$}}
\def\bG{\mbox{\boldmath$ G$}}
\def\bH{\mbox{\boldmath$ H$}}
\def\bI{\mbox{\boldmath$ I$}}
\def\bJ{\mbox{\boldmath$ J$}}
\def\bK{\mbox{\boldmath$ K$}}
\def\bL{\mbox{\boldmath$ L$}}
\def\bM{\mbox{\boldmath$ M$}}
\def\bN{\mbox{\boldmath$ N$}}
\def\bO{\mbox{\boldmath$ O$}}
\def\bP{\mbox{\boldmath$ P$}}
\def\bQ{\mbox{\boldmath$ Q$}}
\def\bR{\mbox{\boldmath$ R$}}
\def\bS{\mbox{\boldmath$ S$}}
\def\bT{\mbox{\boldmath$ T$}}
\def\bU{\mbox{\boldmath$ U$}}
\def\bV{\mbox{\boldmath$ V$}}
\def\bW{\mbox{\boldmath$ W$}}
\def\bX{\mbox{\boldmath$ X$}}
\def\bY{\mbox{\boldmath$ Y$}}
\def\bZ{\mbox{\boldmath$ Z$}}
\def\ba{\mbox{\boldmath$ a$}}
\def\bb{\mbox{\boldmath$ b$}}
\def\bc{\mbox{\boldmath$ c$}}
\def\bd{\mbox{\boldmath$ d$}}
\def\be{\mbox{\boldmath$ e$}}
\def\bff{\mbox{\boldmath$ f$}}
\def\bg{\mbox{\boldmath$ g$}}
\def\bh{\mbox{\boldmath$ h$}}
\def\bi{\mbox{\boldmath$ i$}}
\def\bj{\mbox{\boldmath$ j$}}
\def\bk{\mbox{\boldmath$ k$}}
\def\bl{\mbox{\boldmath$ l$}}
\def\bm{\mbox{\boldmath$ m$}}
\def\bn{\mbox{\boldmath$ n$}}
\def\bo{\mbox{\boldmath$ o$}}
\def\bp{\mbox{\boldmath$ p$}}
\def\bq{\mbox{\boldmath$ q$}}
\def\br{\mbox{\boldmath$ r$}}
\def\bs{\mbox{\boldmath$ s$}}
\def\bt{\mbox{\boldmath$ t$}}
\def\bu{\mbox{\boldmath$ u$}}
\def\bv{\mbox{\boldmath$ v$}}
\def\bw{\mbox{\boldmath$ w$}}
\def\bx{\mbox{\boldmath$ x$}}
\def\by{\mbox{\boldmath$ y$}}
\def\bz{\mbox{\boldmath$ z$}}
%----------------------------------------
\title{\LARGE A Diffuse Interface Framework for Modeling the Evolution of Multi-cell Aggregates as a Soft Packing Problem}
\subtitle{}

\author{S. Rudraraju$^1$ ,J. Jiang$^2$ , D.  Auddya$^1$ ,T. Topal$^2$ ,L. V. Diaz$^3$ , K. Garikipati$^2$
\\ \vspace{0.2in}
$^1$ University of Wisconsin Madison\\ \vspace{0.1in}
$^2$ University of Michigan Ann Arbor\\ \vspace{0.1in}
$^3$ Oakland University, Michigan
 %\includegraphics[height=0.35\textheight]{figures/simulations/reflectedShellZoom2.png}
 }
\date{WCCM 2018\\ \vspace{.10cm} New York,  July 25, 2018}
%\author[The author]{\includegraphics[height=1cm,width=2cm]{figures/reflectedShellZoom.png}\\The Author}

%start the show
\begin{document}

\frame{\titlepage}
 
 %Slide
\frame{
\frametitle{Soft packing problem: Overview}
\begin{itemize} 
\item Motivation \vspace{0.025in}
\begin{itemize} \vspace{0.025in}
\item Embroyogenesis \vspace{0.05in}
\item Tumor growth \vspace{0.025in}
\end{itemize}  \vspace{0.025in}
\item Previous models\vspace{0.025in}
\begin{itemize}
\item Vertex based, cell based, cellular automata. \vspace{0.025in}
\end{itemize}  \vspace{0.025in}
\item Phase field formulation of soft packing \vspace{0.05in}
\item Mechanics of soft packing \vspace{0.05in}
\item Material models \vspace{0.05in}
\item Summary
\end{itemize}
} 

\frame{
  \frametitle{Motivation: Embroyogenesis}
  \begin{figure}[hbt]
       \begin{center}
       \includegraphics[width=0.8\textwidth]{figures2/embroyo2.png}{}
       \end{center}
       \caption*{ Early cleavages of C. subdepressus under light microscopy [Reference: B. C. Vellutini and A. E. Migotto, PLOS One, 2010]}
   \end{figure}
   \vspace{-0.15in}
    \href{run:movies/ASeaBiscuitsLife.mp4}{\small Embroyogenesis in C. subdepressus}
}

\frame{
  \frametitle{Motivation: Embroyogenesis}
  \begin{figure}[hbt]
       \begin{center}
       \includegraphics[width=\textwidth]{figures2/journalpcbi1001128g001.png}{}
       \end{center}
       \caption*{Schematic view of morphological and lineage specification steps during the early mouse embryonic development [Reference: Krupinski P, Chickarmane V, Peterson C (2011),  PLoS Comput Biol 7(5): e1001128]}
   \end{figure}
}

\frame{
  \frametitle{Motivation: Tumor growth}
  \begin{figure}[hbt]
       \begin{center}
       \includegraphics[width=0.8\textwidth]{figures2/tumorMicroEnvironment.png}{}
       \end{center}
       \caption*{Complexity of the tumor microenvironment [Reference: Bumsoo Han et al., Cancer Letters, Vol. 380: 1, 2016]}
   \end{figure}
   \vspace{-0.15in}
    \href{run:movies/tumorGrowth.mp4}{\small Cell packing in growing tumors [Reference: Mills Lab, RPI]}
}

\frame{
  \frametitle{Soft packing of cells in cellular aggregates}
  \begin{figure}[hbt]
       \begin{center}
       \includegraphics[width=\textwidth]{figures2/packing.eps}{}
       \end{center}
   \end{figure}
   \vspace{0.2in}
\small{$^1$ https://commons.wikimedia.org/w/index.php?curid=29251495} \\
\small{$^2$ Embryo of Echinaster brasiliensis (A. E Migotto, Universidade de Sao Paulo) https://www.cell.com/pictureshow/embryogenesis}
}

\frame{
  \frametitle{Relevant numerical models: Cellular automata / High-Q Potts models}
\begin{figure}
    \centering
    \begin{subfigure}[b]{0.45\textwidth}
        \includegraphics[width=\textwidth]{figures2/ElementaryCARule030_700.png}
        \caption*{Cellular automata rules$^1$}
        %\label{fig:gull}
    \end{subfigure}
    ~ %add desired spacing between images, e. g. ~, \quad, \qquad, \hfill etc. 
      %(or a blank line to force the subfigure onto a new line)
    \begin{subfigure}[b]{0.45\textwidth}
        \includegraphics[width=\textwidth]{figures2/CASimulation.png}
        \caption*{Clustering dynamics using CA models$^2$} 
      %\label{fig:tiger}
    \end{subfigure}
  % \caption{Pictures of animals}\label{fig:animals}
\end{figure}
\small{$^1$ http://mathworld.wolfram.com/CellularAutomaton.html} \\
\small{$^2$ Y. Zhang et al., PLoS ONE 6(10): e24999. doi:10.1371/journal.pone.0024999, 2011}
}

\frame{
  \frametitle{Relevant numerical models: Cell and vertex based models}
\begin{figure}
    \centering
    \begin{subfigure}[b]{0.45\textwidth}
        \includegraphics[width=\textwidth]{figures2/ElementaryCARule030_700.png}
        \caption*{Cellular automata rules$^1$}
        %\label{fig:gull}
    \end{subfigure}
    ~ %add desired spacing between images, e. g. ~, \quad, \qquad, \hfill etc. 
      %(or a blank line to force the subfigure onto a new line)
    \begin{subfigure}[b]{0.45\textwidth}
        \includegraphics[width=\textwidth]{figures2/CASimulation.png}
        \caption*{Clustering dynamics using CA models$^2$} 
      %\label{fig:tiger}
    \end{subfigure}
  % \caption{Pictures of animals}\label{fig:animals}
\end{figure}
\small{$^1$ http://mathworld.wolfram.com/CellularAutomaton.html} \\
\small{$^2$ Y. Zhang et al., PLoS ONE 6(10): e24999. doi:10.1371/journal.pone.0024999, 2011}
}

\frame{
  \frametitle{Soft packing: A novel phase field approach}
    \begin{figure}[hbt]
       \begin{center}
       \includegraphics[width=\textwidth]{figures2/phaseField12Cells.png}{}
       \end{center}
   \end{figure}
   %\vspace{-0.15in}
    \href{run:movies/phaseField12Cells.mp4}{\small Phase field simulation of soft packing}
}

%Slide
\frame{
\frametitle{Phase field modeling}
\begin{columns}[T] % align columns
\begin{column}{.47\textwidth}
Cahn-Hilliard dynamics
 \begin{equation*}
\Pi(c, \grad  c) = \int_{\Omega}  \left[  f( c ) + \grad  c  \cdot \kappa(\grad c) \grad  c  \right]  dV 
\end{equation*}
Chemical potential:
\begin{align*}
	   \v{\mu} = \delta_{c}\Pi( c , \grad  c)  
\end{align*}   
Kinetics:
\begin{align*}
   \frac{\partial c }{\partial t} = \grad \cdot (-\bL (\grad c) \grad \v{\mu}) 
\end{align*}  
\begin{itemize}
\item Models evolution of conserved fields like composition.
\item Fourth order PDE with complex anisotropic dependencies.
\end{itemize} 
%
\end{column}%
\hfill%
\begin{column}{.47\textwidth}
Allen-Cahn dynamics
\begin{equation*}
\Pi(\eta_i, \grad  \eta_i) = \int_{\Omega}  \left[  f( \eta_i ) +  \grad  \eta_i  \cdot \kappa (\grad  \eta_i) \grad  \eta_i  \right]  dV 
\end{equation*}
Chemical potential:
\begin{align*}
	   \v{\mu} = \delta_{\eta_i}\Pi( \eta_i, \grad  \eta_i)  
\end{align*} 
Kinetics:
\begin{align*}
   \frac{\partial \eta_i }{\partial t} = - (\bL (\grad \eta_i) ~ \v{\mu}    ) 
\end{align*}  
\begin{itemize}
\item Models evolution of non-conserved fields like structural order parameters. 
\item System of highly coupled second order PDE's.
\end{itemize} 
%
\end{column}%
\end{columns}
\vfill
\small{\textit{van der Waals, Verhandel. Konink. Akad. Westen. Amsterdam, 1893}}\\
\small{\textit{Cahn ~\& Hilliard, J. Chem. Phys., 1958}}
} 

\frame{
  \frametitle{Phase field modeling}
  \begin{figure} 
  \begin{center} \includegraphics[width=0.95\textwidth]{figures2/diffusion.png} \end{center} 
  \end{figure}
  \href{run:movies/diffusion.mp4}{Comparison of Fickian diffusion and higher order diffusion} 
}


\frame{
  \frametitle{Soft packing: A novel phase field approach}
 Let $\Omega\in \mathbb{R}^2$ with a smooth boundary $\partial \Omega$. Scalar fields $c_k, \;k = 1,\dots,N$ with $c_k \in [0,1]$ serve to delineate the interior and exterior of the cell numbered $k$. Here, the interior of cell $k$ is $\omega_k \subset \Omega$, where $\omega_k = \{ \bX \in \Omega\vert c_k(\bX) =1\}$. The exterior is $\Omega\backslash\omega_k$. The free energy density function is built up beginning with the following form: 
\begin{equation*}
\psi_1(c_k) = \alpha c_k^2(c_k-1)^2 + \frac{\kappa}{2}\vert\grad c_k\vert^2
\label{equ:psi1}
\end{equation*}
where the double-well term, $f(c_k) = \alpha c_k^2(c_k-1)^2$, enforces segregation into $\omega_k$ and $\Omega\backslash\omega_k$. \\
The total free energy of the multi-cell aggregate is a functional $\Pi[\bc]$, defined as
\begin{align*}
\Pi[\bc] &:= \int\limits_\Omega \psi(\bc,\nabla c) ~\text{d}V\nonumber\\
&=\int\limits_\Omega \left(\sum_{k=1}^{N} f(c_k) + \sum_{k=1}^{N} \frac{\kappa}{2}\vert\grad c_k\vert^2 + \sum_{l\ne k}\sum_{k=1}^{N}\lambda c_k^2 c_l^2 \right) ~\text{d}V.
\label{equ:energy}
\end{align*}
Here, $\bc = \{c_1,\dots,c_N\}$, and $\lambda$ is a penalty coefficient that enforces repulsion between any two cells $k,l$ thus modelling cell contact.
}

\frame{
  \frametitle{Soft packing: A novel phase field approach}
  Taking the variational derivative with respect to $c_k$ in Equation \eqref{equ:energy} yields
\begin{align*}
\delta \Pi_k[\bc;w] =  & \left.\frac{\text{d}}{\text{d}\epsilon} \int\limits_{\Omega} \sum_{k=1}^{N} \left(f(c_k+\epsilon w) + \frac{\kappa}{2}\vert\grad (c_k+\epsilon w)\vert^2+ \sum_{l\ne k}\lambda (c_k+\epsilon w)^2 c_l^2\right)  ~\text{d}V \right|_{\epsilon=0} \nonumber\\
=&\int\limits_{\Omega} w \left( f^\prime(c_k) -  \kappa \Delta  c_k  + \sum_{l\ne k}2\lambda  c_k c_l^2 \right) ~dV + \int\limits_{\partial \Omega}   w \kappa \grad c_k \cdot \bn   ~dS
\end{align*} 
where $\bn$ is the unit outward normal vector to $\partial \Omega$. The chemical potential of the $k^\text{th}$ cell is identified as,
\begin{equation*}
\mu_k  = f^\prime(c_k) -  \kappa \Delta c_k + \sum_{l\ne k}2\lambda  c_k c_l^2
\label{equ:chemo_potential}
\end{equation*}
\begin{equation*}
\frac{\partial c_k}{\partial t} = -~\grad \cdot (-M\grad \mu_k) + s_k
\label{equ:dynamic}
\end{equation*}
}

\frame{
  \frametitle{Numerical implementation}
\begin{align*}
c^{n+1}_{k} &= c^{n}_{k} + \Delta t (M~\grad \cdot (\grad \mu^{n+1}_{k})+s_{k})\nonumber\\
\text{where}\quad\mu^{n+1}_{k} &= {f^\prime}^ {n+1}(c_k) -  \kappa \Delta c^{n+1}_{k} + \sum_{l\ne k}2\lambda  c_k^{n+1} {c_l^{n+1}}^2
\end{align*}
The weak form of the problem is: Find $c^{n+1}_k \in \mathscr{S} = \{c\in \mathscr{H}^1(\Omega)\vert c = 0\;\text{and}\; \nabla c \cdot\bn = 0 \;\text{on}\; \partial\Omega\}$ such that for any arbitrary variation $w \in \mathscr{V} = \{w\in \mathscr{H}^1(\Omega)\vert w = 0\;\text{and}\; \nabla w \cdot\bn = 0 \;\text{on}\; \partial\Omega\}$ on $c_k$, the following residual equations  are satisfied:
\begin{align*}
\int\limits_{\Omega}   w c_k^{n+1} ~\text{d}V
&= \int\limits_{\Omega}   (w  c_k^{n} - \grad w \cdot \Delta t M \grad \mu_k^{n+1} + w\Delta t s_k ~\text{d}V\nonumber\\
\int\limits_{\Omega}   w  \mu_{k}^{n+1}  ~\text{d}V 
&=\int\limits_{\Omega}  (w  {f^\prime}^{n+1}(c_k) + \grad w \cdot \kappa \grad c_k^{n+1} ~dV + \int\limits_\Omega w \sum_{l\ne k}2\lambda  c_k^{n+1} {c_l^{n+1}}^2)  ~\text{d}V
\end{align*}
}

\frame{
  \frametitle{Results}
  \begin{figure}[h!]
	\centering
	\begin{subfigure}[h!]{0.4\textwidth}
		\includegraphics[width=\textwidth]{figures2/onecell_2}
		\caption{Initial single circular cell}
		\label{fig:onecell}
	\end{subfigure}
	\begin{subfigure}[h!]{0.4\textwidth}
		\includegraphics[width=\textwidth]{figures2/fourcell_2}
		\caption{Progression to four cells}
		\label{fig:fourcell}
	\end{subfigure}
	\begin{subfigure}[h!]{0.4\textwidth}
		\includegraphics[width=\textwidth]{figures2/eightcell_2}
		\caption{Progression to eight cells}
		\label{fig:eightcell}
	\end{subfigure}
	\begin{subfigure}[h!]{0.4\textwidth}
		\includegraphics[width=\textwidth]{figures2/twelvecell_2}
		\caption{Progression to twelve cells.}
		\label{fig:twelvecell}
	\end{subfigure}
\end{figure}
}

\frame{
  \frametitle{Shape model: Results}
}

\frame{
  \frametitle{Shape model: Extension to material models}
}

\frame{
  \frametitle{Soft packing: Connection to embroyogenesis}
}

\frame{
  \frametitle{Soft packing: Connection to tumor growth}
}

\frame{
  \frametitle{Summary and ongoing work}
}

\frame{
  \frametitle{Thanks!!!}
}

 %%%%%%%%%%%%%%%%%%%%%%%%%%%%%%%%%%%%%%%%%%%%%%%%%%%%%%%
 %%%%%%%%%%%%%%%%%%%%%%%%%%%%%%%%%%%%%%%%%%%%%%%%%%%%%%%%
\iffalse

\frame{
  \frametitle{Growth driven by translating/revolving characteristic generating curves}
  \begin{figure}[hbt]
       \begin{center}
       \includegraphics[width=\textwidth]{figures/baseGeometries.eps}{}
       \end{center}
   \end{figure}
}


\frame{
  \frametitle{Structural foundations of shell surface: Mantle and Periostracum}
  \begin{figure}[hbt]
       \begin{center}
          \includegraphics[width=\textwidth]{figures/MandPMacroScale.jpg}{}
       \end{center}
   \end{figure}
}


\frame{
  \frametitle{Distinct growth modes driven by Mantle and Periostracum}
  \begin{figure}[hbt]
       \begin{center}
          \includegraphics[width=0.8\textwidth]{figures/MandPClassification.eps}{}
       \end{center}
   \end{figure}
}


\frame{
  \frametitle{Modeling evolving fronts by accretive growth}
  \centering $\bF=\bF^{e} \bF^{g}$
  \vspace{0.1in}
    \begin{figure}[hbt]
       \begin{center}
          \includegraphics[width=0.4\textwidth]{figures/fronts2.eps}{}
       \end{center}
   \end{figure}
    \href{run:movies/accretiveGrowth.mp4}{\small Accretive growth}
}


\frame{
  \frametitle{Mantle driven growth: Effect of base geometry - Flat plate}
.\\[-7mm]
\begin{tabular}{c@{}c@{}c}
 \includegraphics[width=0.35\textwidth]{figures/simulations/flatPlateZ0p65}
&\includegraphics[width=0.35\textwidth]{figures/simulations/flatPlateZ0}
& \includegraphics[width=0.25\textwidth]{figures/frontFlat.eps} \\[-5mm]
  \href{run:movies/flatPlateZ0p65.mp4}{\textit{0.5t}} &  \href{run:movies/flatPlateZ0.mp4}{\textit{t}} &\\
 \includegraphics[width=0.35\textwidth]{figures/simulations/flatPlateZ2}
&\includegraphics[width=0.35\textwidth]{figures/simulations/flatPlateZ4} &\\[-5mm]
\href{run:movies/flatPlateZ2.mp4}{\textit{2t}}  & \href{run:movies/flatPlateZ4.mp4}{\textit{4t}} &
\end{tabular}
}

\frame{
  \frametitle{Mantle driven growth: Effect of base geometry - Arc}
.\\[-7mm]
\begin{tabular}{c@{}c@{}c}
 \includegraphics[width=0.35\textwidth]{figures/simulations/arcZ0p5}
&\includegraphics[width=0.35\textwidth]{figures/simulations/arcZ0}
& \includegraphics[width=0.15\textwidth]{figures/arcModels.eps} \\[-2mm]
  {0.5t} &  {t} &  \href{run:movies/arc.mp4}{\tiny [\textit{Arc front growth}]}  \\
 \includegraphics[width=0.35\textwidth]{figures/simulations/arcZ2}
&\includegraphics[width=0.35\textwidth]{figures/simulations/arcZ4} &\\[-5mm]
{2t}  & {4t} &
\end{tabular}
}

\frame{
  \frametitle{Mantle driven growth: Effect of base geometry - Spline}
.\\[-7mm]
\centering
\begin{tabular}{c@{}c@{}c}
 \includegraphics[height=0.4\textheight]{figures/simulations/splineZ0p5}
&\includegraphics[height=0.4\textheight]{figures/simulations/splineZ0p9}
& \includegraphics[width=0.25\textwidth]{figures/splineModels.eps} \\[-2mm]
 \href{run:movies/splineZ0p5.mp4}{\textit{0.5t}} &  \href{run:movies/splineZ0p9.mp4}{\textit{t}} &\\
 \includegraphics[height=0.4\textheight]{figures/simulations/splineZ2}
&\includegraphics[height=0.4\textheight]{figures/simulations/splineZ4} &\\[-5mm]
\href{run:movies/splineZ2.mp4}{\textit{2t}}  & \href{run:movies/splineZ4.mp4}{\textit{4t}} &
\end{tabular}
}


\frame{
  \frametitle{Hierarchical buckling}
  \begin{figure}[hbt]
       \begin{center}
          \includegraphics[width=0.95\textwidth]{figures/simulations/hierarchical.eps}{}
       \end{center}
   \end{figure}
}

\frame{
  \frametitle{Folding back}
  \begin{figure}[hbt]
       \begin{center} 
          \includegraphics[width=0.9\textwidth]{figures/simulations/foldingBack.eps}{}
       \end{center}
   \end{figure}
  \hspace{1.5cm}  \href{run:movies/foldback2.mp4}{\small \textit{Distributed growth}} \hspace{1.6cm}  \href{run:movies/foldback1.mp4}{\small \textit{Localized growth}} 
   }

\frame{
  \frametitle{Periostracum driven growth}
\begin{tabular}{c@{}c@{}c}
 \includegraphics[width=0.3\textwidth]{figures/simulations/barrel1}
&\includegraphics[width=0.3\textwidth]{figures/simulations/barrel2}
& \includegraphics[width=0.3\textwidth]{figures/simulations/barrel.eps} \\[-2mm]
 \href{run:movies/barrel1.mp4}{\textit{t}} &  \href{run:movies/barrel2.mp4}{\textit{2t}} &
\end{tabular}
\vspace{0.25in}\\
 \href{run:movies/eigenBarrelNoMesh.mp4}{\small \textit{[Eigen modes]}} 
}

\frame{
  \frametitle{Modeling coupled Mantle and Periostracum modes (ongoing)}
  \begin{figure}[hbt]
       \begin{center}
          \includegraphics[width=0.55\textwidth]{figures/shellOnSolid.eps}{}
       \end{center}
   \end{figure}
}

\frame{
  \frametitle{Modeling coupled Mantle and Periostracum modes (ongoing)}
  Shell kinematics (Kirchhoff-Love): Given a point $\br$ on the mid-surface and tangent base vectors $\ba_{\alpha}$(=$\br_{,\alpha}$), we have\\
 \begin{align*} 
 a_{\alpha \beta} &= \ba_{\alpha} \cdot \ba_{\beta} \qquad \qquad (\text{first fundamental form})\\
 b_{\alpha \beta} &= {\ba_{\alpha,\beta}} \cdot \ba_{3}  \qquad \qquad (\text{second fundamental form})
\end{align*}
Now the membrane strain and curvature change are given by:
 \begin{align*} 
 \varepsilon_{\alpha \beta} &= \frac{1}{2} (a_{\alpha \beta} - \mathring{a}_{\alpha \beta}) \\
  \kappa_{\alpha \beta} &= (\mathring{b}_{\alpha \beta} - b_{\alpha \beta})
\end{align*}
and the Green-Lagrange strain is given by
 \begin{align*} 
 E_{\alpha \beta} &=  \varepsilon_{\alpha \beta}  + \theta^3 \kappa_{\alpha \beta}
\end{align*}
where $\theta^3$ is the thickness coordinate ($-h/2 \le \theta^3 \le h/2 $). The governing equations are obtained by taking variations of the elastic strain energy density, $W(\bE)$.
 }
 
\frame{
  \frametitle{Summary and look ahead}
  \begin{itemize}
  \setlength\itemsep{1em}
  \item Developed a numerical framework for modeling accretive growth in seashells and similar exoskeleton structures.
  \item Classified growth patterns into mantle-driven and periostracum-driven modes. 
  \item Studied deformation modes as function of geometry (curvature) and growth/calcification rate.
  \item Validating/proposing growth patterns which would yield complex shell structures (hierarchical shapes, folding back modes, bareling, etc.)  
  \item Ongoing work on developing a shell on solid framework for modeling coupled mantle and periostracum driven modes.
  \item Identifying common exoskeleton growth patterns across various species of phylum Mollusca.
  \end{itemize}
}

\fi

 
\end{document} 
